\section{FM-Mode}
\subsection{harmonische Schwingung}
Wir untersuchen hier die Frequenzverschiebung bei einem in Frequency-Modulation( \FM ) Modus gesteuertes Atomic-Force-Microscope ( \AFM ). Als erstes nehmen wir hierbei an, dass wir nur eine harmonische Schwingung besitzen.


\begin{figure}[H]
	\begin{tikzpicture}

	\end{tikzpicture}
	\caption{Verhalten des ideaisierten Cantilevers}
\end{figure}
\subsection{Annahmen}
Der Cantilever wir als harmonischer Oszillator betrachtet. Die resultierende Schwingung soll eine harmonische Schwingung sein.\\
Da unsere Aparatur perfekt eingestellt ist, liegt die Phasendifferenz zwischen unserem Antriebssignal und unserer Antwortsignal 90 Grad.

\subsection{Frequenzverschiebung einer einzelnen Schwingung}

Wir idealisieren unsere Schwingungfeder zu einem harmonischen Oszillator:
\begin{equation}
	m \ddot{z} + \frac{m \omega}{Q_{cant}} \dot{z} + k \left( z-z_{drive} - \Delta L \right) = F_{ts} (d+z) \label{federgleichung} 
\end{equation}

Unser Cantilever besitzt nur eine rein harmonische Schwingung und unserer Antrieb besitzt eine Phasendifferenz zu 90 Grad zur Antwort.
\begin{gather}
	z = A \sin ( \omega t) \\ 
	z_{drive} = A_{drive} \sin ( \omega t)
\end{gather}

Wir interessieren uns für die Frequenzverschiebung die durch das Kraftfeld bewirkt wird.
Nach der linearen Antworttheorie ist die Amplitude der Antwort des Cantilevers nur abhängig von der Antriebskraft. Das ist die Kraft die senkrecht zur Schwingung des Cantilevers läuft. Dazu gehört nicht nur unsere von der Maschine ausgeführte Antriebskraft $F_{drive}$, sondern auch die senkrechten Anteile des Kraftfeldes zwischen Spitze und Oberfläche $F_{ts}$.

Die Verstimmung der Antwort(Resonanzfrequenz) von der natürlichen Antwortfrequenz(Eigenfrequenz) sind dementsprechend die Kraftanteile, die nicht-senkrecht, aso parallel zur Antwort stehen, verantwortlich. 
Diese extrahieren wir aus unserem Kraftfeld durch die Aufteilung auf die Fourrierereihe:
\begin{equation}
	\mitlungz{ \dots } = \int \dots A \sin (\omega t) \mathrm{d} t
\end{equation}

Wenn wir diese Operation auf die Federgleichung anwenden, erhalten wir die relevanten Terme:
\begin{equation}
	0
\end{equation}
Daraus können wir die Frequenzverschiebung berechnen:
\begin{equation}
	\Delta f = - \frac{f_0}{A^2 k } \mitlungz{F_{ts}} \label{gl:frequenzverschiebung}
\end{equation}


\subsection{Gewichtungsfunktion bei konservativem Kraftfeld}

Nach Gleichung \ref{gl:frequenzverschiebung} erhalten wir eine zeitliche Mittlung der Kraft. Wir interessieren uns aber für das Kraftfeld. Darum muss eine Mittelung über den Ort erfolgen.
Nach Substitution erhalten wir:
\begin{equation}
	\mitlungz{F_{ts}} = \int F_{ts}(d + z) \frac{1}{\sqrt{A^2 - z^2}} \mathrm{d}t
\end{equation}

Um diese Gleichung zu vereinfachen stellen wir eine Gewichtungsfunktion $g(z)$ auf, sodass gilt:
\begin{equation}
	%\dots = \int F_{ts}(z+z_0) g(z) \mathrm{d} z
\end{equation}

Die Gewichtungsfunktion können wir dann aus den beiden vorherigen Gleichungen bestimmen:
\begin{equation}
	g(z) = \frac{1}{\sqrt{A^2 - z^2}}
\end{equation}


\subsection{Gewichtungsfunktion bei Bremsung}
Obige Gleichung gilt nur für konservative Kraftfelder
